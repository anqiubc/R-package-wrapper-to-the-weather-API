\nonstopmode{}
\documentclass[a4paper]{book}
\usepackage[times,inconsolata,hyper]{Rd}
\usepackage{makeidx}
\usepackage[utf8]{inputenc} % @SET ENCODING@
% \usepackage{graphicx} % @USE GRAPHICX@
\makeindex{}
\begin{document}
\chapter*{}
\begin{center}
{\textbf{\huge Package `weatherapi'}}
\par\bigskip{\large \today}
\end{center}
\inputencoding{utf8}
\ifthenelse{\boolean{Rd@use@hyper}}{\hypersetup{pdftitle = {weatherapi: Package to wrap the Weather API}}}{}\begin{description}
\raggedright{}
\item[Type]\AsIs{Package}
\item[Title]\AsIs{Package to wrap the Weather API}
\item[Version]\AsIs{0.1.0}
\item[Author]\AsIs{c(person(given = ``Evelyn'',
family = ``Sugihermanto'',
role = c(``aut'', ``cre''),
email = ``evelynsugihermanto@yahoo.com''),
person(given = ``Anqi'',
family = ``Li'',
role = c(``aut'', ``cre''),
email = ``lianqi20209@gmail.com''),
person(given = ``Xin'',
family = ``Tian'',
role = c(``aut'',``cre''),
email = ``tjyh111@163.com''),
person(given = ``Val'',
family = ``Veeramani'',
role = c(``aut'',``cre''),
email = ``veeramanival@gmail.com'')}
\item[Maintainer]\AsIs{Evelyn Sugihermanto }\email{evelynsugihermanto@yahoo.com}\AsIs{}
\item[Description]\AsIs{The weatherapi package is a wrapper package for the Weather API.}
\item[License]\AsIs{MIT + file LICENSE}
\item[Encoding]\AsIs{UTF-8}
\item[Imports]\AsIs{httr, jsonlite}
\item[RoxygenNote]\AsIs{7.1.2}
\item[Suggests]\AsIs{rmarkdown, knitr, testthat (>= 3.0.0), devtools, ggplot2,
reshape2, tidyverse}
\item[VignetteBuilder]\AsIs{knitr}
\item[NeedsCompilation]\AsIs{no}
\end{description}
\Rdcontents{\R{} topics documented:}
\inputencoding{utf8}
\HeaderA{api\_key}{Get or set API\_KEY value}{api.Rul.key}
%
\begin{Description}\relax
This package will need a Weather API key residing in the environment
variable \code{API\_KEY}. Set it in the \code{.Renviron} file in the home directory.
See vignette for more details on how to set up the API key.
\end{Description}
%
\begin{Usage}
\begin{verbatim}
api_key(force = FALSE)
\end{verbatim}
\end{Usage}
%
\begin{Arguments}
\begin{ldescription}
\item[\code{force}] A boolean to force set new API key for current environment.
To force set a new API key set it to TRUE, the default value is FALSE.
\end{ldescription}
\end{Arguments}
%
\begin{Value}
A vector containing the API key
\end{Value}
%
\begin{Examples}
\begin{ExampleCode}
# api_key() # To display API Key that was set in the .Renviron file
# api_key(force=TRUE) # To change the API key, see vignette for details
\end{ExampleCode}
\end{Examples}
\inputencoding{utf8}
\HeaderA{data\_request}{get the astronomy data of specific date time and city from weatherapi.com}{data.Rul.request}
%
\begin{Description}\relax
get the astronomy data of specific date time and city from weatherapi.com
\end{Description}
%
\begin{Usage}
\begin{verbatim}
data_request(query)
\end{verbatim}
\end{Usage}
%
\begin{Arguments}
\begin{ldescription}
\item[\code{query}] A list contain information of city, date, and api\_key, represented by q, dt, key, repectively.
\end{ldescription}
\end{Arguments}
%
\begin{Value}
The http get response containing astronomy information.
\end{Value}
%
\begin{Examples}
\begin{ExampleCode}
data_request(list(key="abc", q="London", dt="2021-01-01"))
\end{ExampleCode}
\end{Examples}
\inputencoding{utf8}
\HeaderA{get\_astronomy}{A wrapper function to obtain the the astronomy data of a the desired city and date time from the weatherapi.com.}{get.Rul.astronomy}
%
\begin{Description}\relax
A wrapper function to obtain the the astronomy data of a the desired city and date time from the weatherapi.com.
\end{Description}
%
\begin{Usage}
\begin{verbatim}
get_astronomy(city, date)
\end{verbatim}
\end{Usage}
%
\begin{Arguments}
\begin{ldescription}
\item[\code{city}] A string indicating city (e.g., "London", "Beijing").

\item[\code{date}] A string of the form yyyy-mm-dd indicating date (e.g., "2022-02-10").
\end{ldescription}
\end{Arguments}
%
\begin{Value}
A dataframe with the columns `sunrise`, `sunset`, `moonrise`, `moonset`, `moon\_phase`, and `moon\_illumination`.
\end{Value}
%
\begin{Examples}
\begin{ExampleCode}
get_astronomy("Beijing", "2022-01-10")
\end{ExampleCode}
\end{Examples}
\inputencoding{utf8}
\HeaderA{get\_current\_weather}{Get current weather information}{get.Rul.current.Rul.weather}
%
\begin{Description}\relax
Get the current real-time weather and air-quality information
\end{Description}
%
\begin{Usage}
\begin{verbatim}
get_current_weather(location, air_quality = "yes")
\end{verbatim}
\end{Usage}
%
\begin{Arguments}
\begin{ldescription}
\item[\code{location}] A vector of location

\item[\code{air\_quality}] A string of "yes" or "no" to show air quality
information. Default is show air quality information ("yes").
\end{ldescription}
\end{Arguments}
%
\begin{Value}
A data frame of current weather information (see vignette for details)
\end{Value}
%
\begin{Examples}
\begin{ExampleCode}
get_current_weather("Kelowna")
get_current_weather(c("Kelowna", "Vancouver"), "no")
\end{ExampleCode}
\end{Examples}
\inputencoding{utf8}
\HeaderA{get\_history\_astro\_information}{Get history astronomical information for a specific day}{get.Rul.history.Rul.astro.Rul.information}
%
\begin{Description}\relax
Get history astronomical information for a specific day
\end{Description}
%
\begin{Usage}
\begin{verbatim}
get_history_astro_information(q, dt)
\end{verbatim}
\end{Usage}
%
\begin{Arguments}
\begin{ldescription}
\item[\code{q}] A string containing the location, which can be US Zipcode, UK Postcode, Canada Postalcode, IP address, Latitude/Longitude (decimal degree) or city name.

\item[\code{dt}] A string containing the date for query, which should be within the last 7 days.
\end{ldescription}
\end{Arguments}
%
\begin{Value}
A data frame containing the astronomical information for the day.
\end{Value}
%
\begin{Examples}
\begin{ExampleCode}
get_history_astro_information("London","2022-02-12")
\end{ExampleCode}
\end{Examples}
\inputencoding{utf8}
\HeaderA{get\_history\_daily\_weather}{Get history weather information for a specific day}{get.Rul.history.Rul.daily.Rul.weather}
%
\begin{Description}\relax
Get history weather information for a specific day
\end{Description}
%
\begin{Usage}
\begin{verbatim}
get_history_daily_weather(q, dt)
\end{verbatim}
\end{Usage}
%
\begin{Arguments}
\begin{ldescription}
\item[\code{q}] A string containing the location, which can be US Zipcode, UK Postcode, Canada Postalcode, IP address, Latitude/Longitude (decimal degree) or city name.

\item[\code{dt}] A string containing the date for query, which should be within the last 7 days.
\end{ldescription}
\end{Arguments}
%
\begin{Value}
A data frame containing the daily weather information for the day.
\end{Value}
%
\begin{Examples}
\begin{ExampleCode}
get_history_daily_weather("London","2022-02-12")
\end{ExampleCode}
\end{Examples}
\inputencoding{utf8}
\HeaderA{get\_history\_hourly\_weather}{Get history hourly weather information for a specific hour in a specific day}{get.Rul.history.Rul.hourly.Rul.weather}
%
\begin{Description}\relax
Get history hourly weather information for a specific hour in a specific day
\end{Description}
%
\begin{Usage}
\begin{verbatim}
get_history_hourly_weather(q, dt, h)
\end{verbatim}
\end{Usage}
%
\begin{Arguments}
\begin{ldescription}
\item[\code{q}] A string containing the location, which can be US Zipcode, UK Postcode, Canada Postalcode, IP address, Latitude/Longitude (decimal degree) or city name.

\item[\code{dt}] A string containing the date for query, which should be within the last 7 days.

\item[\code{h}] An integer containing the hour for query, which be in the range of 0 to 23.
\end{ldescription}
\end{Arguments}
%
\begin{Value}
A data frame containing the daily weather information for the hour in the day.
\end{Value}
%
\begin{Examples}
\begin{ExampleCode}
get_history_hourly_weather("London","2022-02-12",4)
\end{ExampleCode}
\end{Examples}
\inputencoding{utf8}
\HeaderA{get\_sports\_events}{Get sports information}{get.Rul.sports.Rul.events}
%
\begin{Description}\relax
Get a list of all upcoming sports events in a location
\end{Description}
%
\begin{Usage}
\begin{verbatim}
get_sports_events(location, sport)
\end{verbatim}
\end{Usage}
%
\begin{Arguments}
\begin{ldescription}
\item[\code{location}] A vector of location

\item[\code{sport}] A string of sport type ("football","cricket","golf")
\end{ldescription}
\end{Arguments}
%
\begin{Value}
A data frame of upcoming sports events in a location (see vignette for details)
\end{Value}
%
\begin{Examples}
\begin{ExampleCode}
get_sports_events("London", "football")
get_sports_events(c("London", "Oxford"), "football")
\end{ExampleCode}
\end{Examples}
\inputencoding{utf8}
\HeaderA{get\_time\_zone}{Get time zone information}{get.Rul.time.Rul.zone}
%
\begin{Description}\relax
Get the local time-zone information for a location
\end{Description}
%
\begin{Usage}
\begin{verbatim}
get_time_zone(location)
\end{verbatim}
\end{Usage}
%
\begin{Arguments}
\begin{ldescription}
\item[\code{location}] A vector of location
\end{ldescription}
\end{Arguments}
%
\begin{Value}
A data frame of local time-zone information (see vignette for details)
\end{Value}
%
\begin{Examples}
\begin{ExampleCode}
get_time_zone("Kelowna")
get_time_zone(c("Kelowna", "Vancouver"))
\end{ExampleCode}
\end{Examples}
\printindex{}
\end{document}
